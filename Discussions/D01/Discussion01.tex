%% LyX 2.1.4 created this file.  For more info, see http://www.lyx.org/.
%% Do not edit unless you really know what you are doing.
\documentclass{beamer}
\usepackage{hyperref}
\usepackage{minted}
\usepackage{animate}
\usepackage{graphicx}
\def\Put(#1,#2)#3{\leavevmode\makebox(0,0){\put(#1,#2){#3}}}
\usepackage{color}
\usepackage{tikz}
\usepackage{amssymb}
\usepackage{enumerate}

\newcommand\blfootnote[1]{%
  \begingroup
  \renewcommand\thefootnote{}\footnote{#1}%
  \addtocounter{footnote}{-1}%
  \endgroup
}

\definecolor{LightGray}{gray}{0.9}

\ifx\hypersetup\undefined
  \AtBeginDocument{%
    \hypersetup{unicode=true,
 bookmarksnumbered=false,bookmarksopen=false,
 breaklinks=false,pdfborder={0 0 0},colorlinks=false}
  }
\else
  \hypersetup{unicode=true,
 bookmarksnumbered=false,bookmarksopen=false,
 breaklinks=false,pdfborder={0 0 0},colorlinks=false}
\fi

\makeatletter
%%%%%%%%%%%%%%%%%%%%%%%%%%%%%% Textclass specific LaTeX commands.
 % this default might be overridden by plain title style
 \newcommand\makebeamertitle{\frame{\maketitle}}%
 % (ERT) argument for the TOC
 \AtBeginDocument{%
   \let\origtableofcontents=\tableofcontents
   \def\tableofcontents{\@ifnextchar[{\origtableofcontents}{\gobbletableofcontents}}
   \def\gobbletableofcontents#1{\origtableofcontents}
 }

%%%%%%%%%%%%%%%%%%%%%%%%%%%%%% User specified LaTeX commands.
\usetheme{Malmoe}
% or ...
\useoutertheme{infolines}
\addtobeamertemplate{headline}{}{\vskip2pt}

\setbeamercovered{transparent}
% or whatever (possibly just delete it)



\setbeamertemplate{footline}
{
  \leavevmode%
  \hbox{%
  \begin{beamercolorbox}[wd=.5\paperwidth,ht=2.25ex,dp=1ex,center]{title in head/foot}%
    \usebeamerfont{title in head/foot}\insertshorttitle
  \end{beamercolorbox}%
  \begin{beamercolorbox}[wd=.5\paperwidth,ht=2.25ex,dp=1ex,right]{date in head/foot}%
    \usebeamerfont{date in head/foot}\insertshortdate{}\hspace*{2em}
    \insertframenumber{} / \inserttotalframenumber\hspace*{2ex} 
  \end{beamercolorbox}}%
  \vskip0pt%
}

\makeatother

\begin{document}

\title[Discussion 1 - Review]{CS/MATH 111, Discrete Structures - Fall 2018. \\ Discussion 1 - Review }
\author{Andres Calderon}
\institute{University of California, Riverside}

\makebeamertitle

% \AtBeginSection[]{
%   \frame<beamer>{ 
%     \frametitle{Agenda}   
%     \tableofcontents[currentsubsection] 
%   }
% }

\newif\iflattersubsect

\AtBeginSection[] {
    \begin{frame}<beamer>
    \frametitle{Outline} %
    \tableofcontents[currentsection]  
    \end{frame}
    \lattersubsectfalse
}

\AtBeginSubsection[] {
    % \iflattersubsect
    \begin{frame}<beamer>
    \frametitle{Outline} %
    \tableofcontents[currentsubsection]  
    \end{frame}
    % \fi
    % \lattersubsecttrue
}

\section{Logic}

\begin{frame}{Logic}
  For sentences (a) and (b), tell which of the sentences (i)-(v) is \textbf{equivalent} to it.\\
  \begin{enumerate}[(a)]
    \item \textbf{``If X is green or pink, then X is a vegetable.''}
    \begin{enumerate}[(i)]
      \item ``X is green and X is a vegetable.''
      \item ``X is not green and X is not a vegetable.''
      \item ``X is not green or X is a vegetable.''
      \item ``X is not green and X is a vegetable.''
      \item None of the above.
    \end{enumerate}
  \end{enumerate}
\end{frame}

\begin{frame}{Logic}
  For sentences (a) and (b), tell which of the sentences (i)-(v) is \textbf{equivalent} to it.\\
  \begin{enumerate}[(a)]
    \item \textbf{``If X is green or pink, then X is a vegetable.''}
    \begin{enumerate}[(i)]
      \item ``X is green and X is a vegetable.''
      \item ``X is not green and X is not a vegetable.''
      \item ``X is not green or X is a vegetable.''
      \item ``X is not green and X is a vegetable.''
      \item \textbf{None of the above.}
    \end{enumerate}
  \end{enumerate}
\end{frame}

\begin{frame}{Logic}
  For sentences (a) and (b), tell which of the sentences (i)-(v) is \textbf{equivalent} to it.\\
  \begin{enumerate}[(b)]
    \item \textbf{``X is a pig, and Y or Z is a bird.''}
    \begin{enumerate}[(i)]
      \item ``Either X is not a pig, or Y and Z are not birds.''
      \item ``Either X is a pig and Y is a bird, or X is a pig and Z is a bird.''
      \item ``X is not a pig, and neither Y nor Z is a bird.''
      \item ``X is a pig, and both Y and Z are birds.''
      \item None of the above.
    \end{enumerate}
  \end{enumerate}
\end{frame}

\begin{frame}{Logic}
  For sentences (a) and (b), tell which of the sentences (i)-(v) is \textbf{equivalent} to it.\\
  \begin{enumerate}[(b)]
    \item \textbf{``X is a pig, and Y or Z is a bird.''}
    \begin{enumerate}[(i)]
      \item ``Either X is not a pig, or Y and Z are not birds.''
      \item \textbf{``Either X is a pig and Y is a bird, or X is a pig and Z is a bird.''}
      \item ``X is not a pig, and neither Y nor Z is a bird.''
      \item ``X is a pig, and both Y and Z are birds.''
      \item None of the above.
    \end{enumerate}
  \end{enumerate}
\end{frame}

\begin{frame}{Logic}
  For sentences (c), (d), and (e), tell which of the sentences (i)-(v) is its \textbf{negation}.\\
  \begin{enumerate}[(c)]
    \item \textbf{``$\boldsymbol{\forall x\ \exists y : y < x^2 + 17}$''}
    \begin{enumerate}[(i)]
    \item ``$\forall x\ \exists y : y \geq x^2 + 17$''
    \item ``$\forall y\ \exists x : x^2 + 17 < y$''
    \item ``$\exists x\ \exists y : y > x^2 + 17$''
    \item ``$\exists x\ \forall y : y \geq x^2 + 17$''
    \item None of the above.
    \end{enumerate}
  \end{enumerate}
\end{frame}

\begin{frame}{Logic}
  For sentences (c), (d), and (e), tell which of the sentences (i)-(v) is its \textbf{negation}.\\
  \begin{enumerate}[(c)]
    \item \textbf{``$\boldsymbol{\forall x\ \exists y : y < x^2 + 17}$''}
    \begin{enumerate}[(i)]
    \item ``$\forall x\ \exists y : y \geq x^2 + 17$''
    \item ``$\forall y\ \exists x : x^2 + 17 < y$''
    \item ``$\exists x\ \exists y : y > x^2 + 17$''
    \item \textbf{``$\boldsymbol{\exists x\ \forall y : y \geq x^2 + 17}$''}
    \item None of the above.
    \end{enumerate}
  \end{enumerate}
\end{frame}

\section{Counting}

\begin{frame}{Counting}
  \begin{itemize}
    \item Anything that could move (or could be moved), will be tracked...
    \begin{itemize}
      \item Sensors, Sensors everywhere...
      \item Smart phones, GPS, RFID, WiFi, Bluetooth, ZigBee, IoT, Satellites, Drones, Surveillance cameras...
    \end{itemize}
  \end{itemize}
\end{frame}

\section{Equations}

\subsection{Quadratic Equations}

\begin{frame}{Quadratic equations}
  \begin{itemize}
    \item Anything that could move (or could be moved), will be tracked...
    \begin{itemize}
      \item Sensors, Sensors everywhere...
      \item Smart phones, GPS, RFID, WiFi, Bluetooth, ZigBee, IoT, Satellites, Drones, Surveillance cameras...
    \end{itemize}
  \end{itemize}
\end{frame}

\subsection{Cubic Equations}

\begin{frame}{Cubic equations}
  \begin{itemize}
    \item Anything that could move (or could be moved), will be tracked...
    \begin{itemize}
      \item Sensors, Sensors everywhere...
      \item Smart phones, GPS, RFID, WiFi, Bluetooth, ZigBee, IoT, Satellites, Drones, Surveillance cameras...
    \end{itemize}
  \end{itemize}
\end{frame}

%\begin{frame}{Applications}
%  Visual mining of cyclone paths.
%  \centering
%  \includegraphics[width=.85\linewidth]{Figures/cyclones.png}
%  \blfootnote{\tiny \href{https://tinyurl.com/nhmwerr}{(Turdukulov, IJGIS'14)}}
%\end{frame}

\
\end{document}
