\documentclass{beamer}

\usefonttheme{professionalfonts} % using non standard fonts for beamer
\usefonttheme{serif} % default family is serif

\usepackage{hyperref}

\usepackage{animate}

\usepackage{graphicx}

\def\Put(#1,#2)#3{\leavevmode\makebox(0,0){\put(#1,#2){#3}}}

\usepackage{color}

\usepackage{tikz}

\usepackage{amssymb}

\usepackage{enumerate}

\usepackage{minted}

\newcommand\blfootnote[1]{%

  \begingroup

  \renewcommand\thefootnote{}\footnote{#1}%

  \addtocounter{footnote}{-1}%

  \endgroup

}

\makeatletter

%%%%%%%%%%%%%%%%%%%%%%%%%%%%%% Textclass specific LaTeX commands.

 % this default might be overridden by plain title style

 \newcommand\makebeamertitle{\frame{\maketitle}}%

 % (ERT) argument for the TOC

 \AtBeginDocument{%

   \let\origtableofcontents=\tableofcontents

   \def\tableofcontents{\@ifnextchar[{\origtableofcontents}{\gobbletableofcontents}}

   \def\gobbletableofcontents#1{\origtableofcontents}

 }

%%%%%%%%%%%%%%%%%%%%%%%%%%%%%% User specified LaTeX commands.

\usetheme{Malmoe}

% or ...

\useoutertheme{infolines}

\addtobeamertemplate{headline}{}{\vskip2pt}



\setbeamercovered{transparent}

% or whatever (possibly just delete it)

\makeatother

\begin{document}
\title[Discussion 6 - Non-homogeneous Recurrences,  Divide and Conquer \& Inclusion - Exclusion]{CS/MATH 111, Discrete Structures - Fall 2018. \\ Discussion 6 - Non-homogeneous Recurrences,  Divide and Conquer \& Inclusion - Exclusion }
\author[CS111]{Andres, Sara, Elena}
\institute[Fall'18]{University of California, Riverside}
\makebeamertitle
\newif\iflattersubsect

\AtBeginSection[] {
    \begin{frame}<beamer>
    \frametitle{Outline} 
    \tableofcontents[currentsection]  
    \end{frame}
    \lattersubsectfalse
}

\AtBeginSubsection[] {
    \begin{frame}<beamer>
    \frametitle{Outline} 
    \tableofcontents[currentsubsection]  
    \end{frame}
}

\section{Non-homogeneous recurrence}

\begin{frame}{Non-homogeneous recurrence\footnote{\scriptsize Proof available at [Rosen, 2015. pg 521].}}
    \begin{theorem}\label{theo:1}
    {\Large $$f_n = f_n^{'} + f_n^{''}$$}
    
    If $\{f_n^{''}\}$ is a particular solution of the non-homogeneous linear recurrence relation with constant coefficients: 
    $$ f_n = c_1 \cdot f_{n-1} + c_2 \cdot f_{n-2} + \cdots + c_k \cdot f_{n-k} + g(n) $$
    then every solution is of the form $\{f_n^{'} + f_n^{''}\}$, where $\{f_n^{'}\}$ is a solution of the associated homogeneous recurrence relation.
    \end{theorem}
\end{frame}

\begin{frame}{Non-homogeneous recurrence}
    \centering
    \includegraphics[width=.75\linewidth]{r.png}
\end{frame}

\begin{frame}{Non-homogeneous recurrence}
    \begin{itemize}
        \item Find a particular solution for recurrence relation:
            \begin{equation}\tag{1}
                f_n = 3 \cdot f_{n-1} + f_{n-2} + 6
            \end{equation}
    \end{itemize}
\end{frame}

\begin{frame}{Non-homogeneous recurrence}
    \begin{itemize}
        \item Find a particular solution for recurrence relation:
            \begin{equation}\tag{1}
                f_n = 3 \cdot f_{n-1} + f_{n-2} + 6
            \end{equation}
        \item $g(n) = 6$, so:
            \begin{equation}\tag{2}
                f_n^{''} = p_0
            \end{equation}
    \end{itemize}
\end{frame}

\begin{frame}{Non-homogeneous recurrence}
    \begin{itemize}
        \item Find a particular solution for recurrence relation:
            \begin{equation}\tag{1}
                f_n = 3 \cdot f_{n-1} + f_{n-2} + 6
            \end{equation}
        \item $g(n) = 6$, so:
            \begin{equation}\tag{2}
                f_n^{''} = p_0
            \end{equation}
        \item Plugin $(2)$ in $(1)$ becomes: 
            $$ p_0 = 3 \cdot p_0 + p_0 + 6 $$
            $$ p_0 - p_0 - 3 \cdot p_0 = 6 $$
            \begin{equation}\tag{3}
                p_0 = - \frac{6}{2} =  -2
            \end{equation}
    \end{itemize}
\end{frame}

\begin{frame}{Non-homogeneous recurrence}
    \begin{itemize}
        \item Find a particular solution for recurrence relation:
            \begin{equation}\tag{1}
                f_n = 3 \cdot f_{n-1} + f_{n-2} + 6
            \end{equation}
        \item $g(n) = 6$, so:
            \begin{equation}\tag{2}
                f_n^{''} = p_0
            \end{equation}
        \item Plugin $(2)$ in $(1)$ becomes: 
            $$ p_0 = 3 \cdot p_0 + p_0 + 6 $$
            $$ p_0 - p_0 - 3 \cdot p_0 = 6 $$
            \begin{equation}\tag{3}
                p_0 = - \frac{6}{2} =  -2
            \end{equation}
        \item Finally, $(3)$ in $(2)$:
        $$ f_n^{''} = -2 $$
    \end{itemize}
\end{frame}

\begin{frame}{Non-homogeneous recurrence}
   \begin{itemize}
        \item Find a particular solution for recurrence relation:
            \begin{equation}\tag{1}
                f_n = 3 \cdot f_{n-1} + f_{n-2} + 3 \cdot 2^n
            \end{equation}
      \end{itemize}
\end{frame}

\begin{frame}{Non-homogeneous recurrence}
   \begin{itemize}
        \item Find a particular solution for recurrence relation:
            \begin{equation}\tag{1}
                f_n = 3 \cdot f_{n-1} + f_{n-2} + 3 \cdot 2^n
            \end{equation}
        \item $g(n) = 3 \cdot 2^n$, so:
            \begin{equation}\tag{2}
                f_n^{''} = p_0 \cdot 2^n
            \end{equation}
      \end{itemize}
\end{frame}

\begin{frame}{Non-homogeneous recurrence}
   \begin{itemize}
        \item Plug $(2)$ in $(1)$ becomes:
            $$ p_0 \cdot 2^n = 3 \cdot p_0 \cdot 2^{n-1} + p_0 \cdot 2^{n-2} + 3 \cdot 2^n $$
            $$ p_0 \cdot 2^n = 2^{n-2}( 3 \cdot p_0 \cdot 2^{1} + p_0 \cdot 2^{0} + 3 \cdot 2^2) $$
            $$ p_0 \cdot 2^2 = 3 \cdot p_0 \cdot 2 + p_0 + 3 \cdot 4 $$
            $$ p_0 \cdot 4 = 7 \cdot p_0 + 12 $$
            \begin{equation}\tag{3}
                p_0 = -4
            \end{equation}
      \end{itemize}
\end{frame}

\begin{frame}{Non-homogeneous recurrence}
   \begin{itemize}
        \item Plug $(2)$ in $(1)$ becomes:
            $$ p_0 \cdot 2^n = 3 \cdot p_0 \cdot 2^{n-1} + p_0 \cdot 2^{n-2} + 3 \cdot 2^n $$
            $$ p_0 \cdot 2^n = 2^{n-2}( 3 \cdot p_0 \cdot 2^{1} + p_0 \cdot 2^{0} + 3 \cdot 2^2) $$
            $$ p_0 \cdot 2^2 = 3 \cdot p_0 \cdot 2 + p_0 + 3 \cdot 4 $$
            $$ p_0 \cdot 4 = 7 \cdot p_0 + 12 $$
            \begin{equation}\tag{3}
                p_0 = -4
            \end{equation}
        \item Finally, $(3)$ in $(2)$:
            $$ f_n^{''} = -4 \cdot 2^n $$
      \end{itemize}
\end{frame}

\begin{frame}{Non-homogeneous recurrence}
    Solve the following non-homogeneous recurrence:
    \begin{equation}\tag{1}
        f_n = 4 \cdot f_{n-1} - 4 \cdot f_{n-2} + 2 \cdot 5^n
    \end{equation}
    with initial condition: $f_0 = 1$ and $f_1 = 2$.
\end{frame}

\begin{frame}{Non-homogeneous recurrence}
    Solve the following non-homogeneous recurrence:
    \begin{equation}\tag{1}
        f_n = 4 \cdot f_{n-1} - 4 \cdot f_{n-2} + 2 \cdot 5^n
    \end{equation}
    with initial condition: $f_0 = 1$ and $f_1 = 2$.
    
    \begin{itemize}
        \item $ f_n^{'} =  4 \cdot f_{n-1} - 4 \cdot f_{n-2}$
        \begin{enumerate}
            \item Caractheristic equations and its roots:
                $$ x^2 - 4 \cdot x + 4 = 0 $$
                $$ (x - 2)(x - 2) = 0 $$
                $$ x_{1,2} = 2 $$
            \item General form of the solution:
                $$ f_n^{'} = \alpha_1 \cdot 2^n + \alpha_2 \cdot n \cdot 2^n $$
        \end{enumerate}
    \end{itemize}
\end{frame}

\begin{frame}{Non-homogeneous recurrence}
    Solve the following non-homogeneous recurrence:
    \begin{equation}\tag{1}
        f_n = 4 \cdot f_{n-1} - 4 \cdot f_{n-2} + 2 \cdot 5^n
    \end{equation}
    with initial condition: $f_0 = 1$ and $f_1 = 2$.
    
    \begin{itemize}
        \item $ g(n) =  2 \cdot 5^n$, so:
            \begin{equation}\tag{2}
                f_n^{''} = p_0 \cdot 5^n
            \end{equation}
        \item Plug $(2)$ in $(1)$ becomes:
            $$ p_0 \cdot 5^n = 4 \cdot p_0 \cdot 5^{n-1} - 4 \cdot p_0 \cdot 5^{n-2} + 2 \cdot 5^n $$
            \begin{equation}\tag{3}
                p_0 = \frac{50}{9}
            \end{equation}
        \item Finally, $(3)$ in $(2)$:
            $$ f_n^{''} = \frac{50}{9} \cdot 5^n $$
    \end{itemize}
\end{frame}

\begin{frame}{Non-homogeneous recurrence}
    Solve the following non-homogeneous recurrence:
    \begin{equation}\tag{1}
        f_n = 4 \cdot f_{n-1} - 4 \cdot f_{n-2} + 2 \cdot 5^n
    \end{equation}
    with initial condition: $f_0 = 1$ and $f_1 = 2$.
    
    \begin{itemize}
        \item $ f_n^{'} = \alpha_1 \cdot 2^n + \alpha_2 \cdot n \cdot 2^n $
        \item $ f_n^{''} = \frac{50}{9} \cdot 5^n $
        \item $ f_n = \alpha_1 \cdot 2^n + \alpha_2 \cdot n \cdot 2^n + \frac{50}{9} \cdot 5^n $
        \begin{enumerate}[3]
            \item Initial condition equations and their solutions:
                \begin{tabular}{l l l}
                    $f_0 = $ & $\alpha_1 \cdot 2^0 + \alpha_2 \cdot 0 \cdot 2^0 + \frac{50}{9} \cdot 5^0$ & $= \alpha_1 + \frac{50}{9} = 1$ \\
                    $f_1 = $ & $\alpha_1 \cdot 2^1 + \alpha_2 \cdot 1 \cdot 2^1 + \frac{50}{9} \cdot 5^1$ & $= 2 \cdot \alpha_1 + 2 \cdot \alpha_2 + 5 \cdot \frac{50}{9} = 2 $ \\
                \end{tabular}
                where $\alpha_1 = -\frac{41}{9} $ and $ \alpha_2 = -\frac{25}{3}$.
            \item[4] Final answer:
                $$ \vdots $$
        \end{enumerate}
    \end{itemize}
\end{frame}

\section{Divide and Conquer}

\begin{frame}{Divide and Conquer}
    \centering
    \begin{theorem}
        Let $a \geq 0$, $b > 0$, $c > 0$ and $d \geq 0$.  If $T(n)$ satisfies the recurrence then
        $$ T(n) = a \cdot T \Big(\frac{n}{b}\Big) + c \cdot n^d $$
    \end{theorem}
    
    $T(n) =$ 
    \left\{
    \begin{tabular}{l r r}
        $\Theta (n^{\log_b a})$ & \hspace{1cm} & $a > b^d$ \\
        $\Theta (n^d \log n)$   & \hspace{1cm} & $a = b^d$ \\
        $\Theta (n^d)$          & \hspace{1cm} & $a < b^d$ 
    \end{tabular}
\end{frame}

\begin{frame}{Divide and Conquer}
    \centering
    \begin{itemize}
        \item Give the asymptotic value (using the $\Theta$-notation) for the number of letters that will be printed by the following algorithms.  You need to provide an appropriate recurrence equation and its solution.
    \end{itemize}
\end{frame}

\begin{frame}[fragile]{Divide and Conquer}
    \begin{center}
    \begin{minipage}{0.5\textwidth}
    \begin{minted}[fontsize=\scriptsize, 
		   linenos,
		   numbersep=1pt,
		   frame=lines,
		   framesep=3mm]{scala}
def PrintXs(n: integer)
    if(n < 3)
        print("X")
    else
        PrintXs(n/3)
        PrintXs(n/3)
        PrintXs(n/3)
        for(i <- 1 to 2*n)
            print("X")
    \end{minted}
    \end{minipage}
    \end{center}
\end{frame}

\begin{frame}[fragile]{Divide and Conquer}
    \begin{center}
    \begin{minipage}{0.5\textwidth}
    \begin{minted}[fontsize=\scriptsize, 
		   linenos,
		   numbersep=1pt,
		   frame=lines,
		   framesep=3mm]{scala}
def PrintXs(n: integer)
    if(n < 3)
        print("X")
    else
        PrintXs(n/3)
        PrintXs(n/3)
        PrintXs(n/3)
        for(i <- 1 to 2*n)
            print("X")
    \end{minted}
    \end{minipage}
    \end{center}
    
    \begin{itemize}
        \item We have 3 recursive calls, each with parameter $\frac{n}{3}$.
        \item Recurrence is $X(n) = 3 \cdot X \Big(\frac{n}{3}\Big) + 2 \cdot n$.
        \item If $a = 3$, $b = 3$, $c = 2$ and $d = 1$ then $a = b^d$ and $\Theta (n^d \log n)$.
    \end{itemize}
\end{frame}

\begin{frame}[fragile]{Divide and Conquer}
    \begin{center}
    \begin{minipage}{0.5\textwidth}
    \begin{minted}[fontsize=\scriptsize, 
		   linenos,
		   numbersep=1pt,
		   frame=lines,
		   framesep=3mm]{scala}
def PrintYs(n: integer)
    if(n < 2)
        print("Y")
    else
        for(i <- 1 to 16)
            PrintYs(n/2)
        for(i <- 1 to n^3)
            print("Y")
    \end{minted}
    \end{minipage}
    \end{center}
\end{frame}

\begin{frame}[fragile]{Divide and Conquer}
    \begin{center}
    \begin{minipage}{0.5\textwidth}
    \begin{minted}[fontsize=\scriptsize, 
		   linenos,
		   numbersep=1pt,
		   frame=lines,
		   framesep=3mm]{scala}
def PrintYs(n: integer)
    if(n < 2)
        print("Y")
    else
        for(i <- 1 to 16)
            PrintYs(n/2)
        for(i <- 1 to n^3)
            print("Y")
    \end{minted}
    \end{minipage}
    \end{center}
    
    \begin{itemize}
        \item We have 16 recursive calls, each with parameter $\frac{n}{2}$.
        \item Recurrence is $X(n) = 16 \cdot X \Big(\frac{n}{2}\Big) + n^3$.
        \item If $a = 16$, $b = 2$, $c = 1$ and $d = 3$ then $a > b^d$ and $\Theta (n^{\log_2 16})$.
    \end{itemize}
\end{frame}

\begin{frame}[fragile]{Divide and Conquer}
    \begin{center}
    \begin{minipage}{0.5\textwidth}
    \begin{minted}[fontsize=\scriptsize, 
		   linenos,
		   numbersep=1pt,
		   frame=lines,
		   framesep=3mm]{scala}
def PrintZs(n: integer)
    if(n < 3)
        print("Z")
    else
        PrintZs(n/3)
        PrintZs(n/3)
        for(i <- 1 to 7*n)
            print("Z")
    \end{minted}
    \end{minipage}
    \end{center}
\end{frame}

\begin{frame}[fragile]{Divide and Conquer}
    \begin{center}
    \begin{minipage}{0.5\textwidth}
    \begin{minted}[fontsize=\scriptsize, 
		   linenos,
		   numbersep=1pt,
		   frame=lines,
		   framesep=3mm]{scala}
def PrintZs(n: integer)
    if(n < 3)
        print("Z")
    else
        PrintZs(n/3)
        PrintZs(n/3)
        for(i <- 1 to 7*n)
            print("Z")
    \end{minted}
    \end{minipage}
    \end{center}
    \begin{itemize}
        \item We have 2 recursive calls, each with parameter $\frac{n}{3}$.
        \item Recurrence is $X(n) = 2 \cdot X \Big(\frac{n}{3}\Big) + 7 \cdot n$.
        \item If $a = 2$, $b = 3$, $c = 7$ and $d = 1$ then $a < b^d$ and $\Theta (n)$.
    \end{itemize}    
\end{frame}

\begin{frame}{Divide and Conquer}
    \centering
    \includegraphics[trim={0 0 0 2.5cm}, clip, width=.375\linewidth]{a1.PNG}
    \includegraphics[width=.375\linewidth]{a2.PNG}
\end{frame}

\begin{frame}{Divide and Conquer}
    \centering
    \includegraphics[width=.7\linewidth]{e.PNG}
\end{frame}

\section{Inclusion-Exclusion }

\begin{frame}{Inclusion-Exclusion}
    \centering
    \includegraphics[width=.7\linewidth]{3.PNG}
\end{frame}
\begin{frame}{Inclusion-Exclusion}
    \centering
    \includegraphics[width=.7\linewidth]{4.PNG}
\end{frame}

\begin{frame}{Inclusion-Exclusion}
    \centering
    \begin{tabular}{r c l}
    Us citizens:                & $\mid A \mid$                & $= \frac{X}{2}$    \\
    Mexican citizen:            & $\mid B \mid$                & $= 10$             \\
    Canadian citizen:           & $\mid C \mid$                & $= 17$             \\
    US-Mexican citizen:         & $\mid A \cap B \mid$         & $= 4$              \\
    US-Canadian citizen:        & $\mid A \cap C \mid$         & $= 5$              \\
    Canadian-Mexican:           & $\mid B \cap C \mid$         & $= 6$              \\
    Citizens of all countries:  & $\mid A \cap B \cap C \mid$  & $= 2$              \\
    \end{tabular}

    $$X = \frac{X}{2} + 10 + 17 - 4 - 5 - 6 + 2$$
    $$X = \frac{X}{2} + 14$$
    $$X = 28$$

\end{frame}


\end{document}
